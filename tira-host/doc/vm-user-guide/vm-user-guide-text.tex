\section{Welcome}

Welcome to TIRA. TIRA is the first web service to facilitate software submissions at shared tasks. From a scientific viewpoint, software submissions are attractive since they allow anyone to re-evaluate and compare a task's solution on any new or rearranged datasets in the future (given your consent).

To make software submissions convenient and transparent, TIRA provides each user with a personal, fully accessible virtual machine on our servers. We ask you to deploy your software into this virtual machine. The purpose of this user guide is to describe how to access your virtual machine and how to make a software submission.


\section{Getting a Virtual Machine}

Once you have registered for your shared task, you will receive a welcome e-mail which asks for operating system preferences (e.g., Ubuntu or Windows). After a corresponding virtual machine has been set up for you, you will receive another e-mail with all the necessary information to access it (\texttt{XX} refers to a two-digit number):\\
\begin{center}
\begin{tabular}{@{}lr@{}}
\toprule
Host address & webisXX.medien.uni-weimar.de \\  
User name& \texttt{<name>} \\  
User password & \texttt{<password>} \\  
SSH port & 444\texttt{XX} \\
RDP port & 555\texttt{XX} \\ 
\bottomrule
\end{tabular}
\end{center}

\section{Accessing your Virtual Machine}

You can access your virtual machine via SSH and via remote desktop. The following sections explain how to access them, dependent on the operating system you are using.

\subsection*{From Ubuntu to a Ubuntu Virtual Machine}

\enlargethispage{\baselineskip}
\medskip\noindent
Use the following command to access your virtual machine via SSH:\\
\texttt{ssh <name>@<host-address> -p <ssh-port>}

\medskip\noindent
Use the following command to access your virtual machine via remote desktop~(RDP):\\
\texttt{rdesktop <host-address>:<rdp-port> -u <name> -p <password>}

\subsection*{From Windows to a Windows Virtual machine}

In order to access a Windows virtual machine via SSH, you first need to install an SSH client on your~PC. We recommend to use OpenSSH which you can download from Sourceforge.%
\footnote{http://sshwindows.sourceforge.net/download/}
After installing OpenSSH use the following command to access your virtual machine via~SSH:\\
\texttt{ssh <name>@<host-address> -p <ssh-port>}

\medskip\noindent
In order to access a Windows virtual machine via Remote Desktop, we recommend you use the Windows Remote Desktop Connection Dialog.

\paragraph{Step 1}

Press the Windows Start button. Enter ``remote desktop connection'' into the search field. Start Remote Desktop Connection Dialog which is listed in the category Programs.

\begin{figure}[H]
\centering
\includegraphics[scale=0.7]{images/1}
\label{fig1}
\end{figure}

\paragraph{Step 2}

Press Options.
\begin{figure}[H]
\centering
\includegraphics[scale=0.7]{images/2}
\label{fig2}
\end{figure}

\pagebreak
\paragraph{Step 3}

Enter your virtual machine access information:\\
\indent Computer: \texttt{<host-address>:<rdp-port>}\\
\indent User Name: \texttt{<name>} \\
Check ``Allow me to save credentials'' (which is very important!). Press Connect.

\begin{figure}[H]
\centering
\includegraphics[scale=0.7]{images/3}
\label{fig3}
\end{figure}

\paragraph{Step 4}

Enter your \texttt{<password>} and press OK.

\begin{figure}[H]
\centering
\includegraphics[scale=0.7]{images/4}
\label{fig4}
\end{figure}

\section{Permissions on the Virtual Machine}

Your user is member of the sudoer group (Ubuntu) or the admin group (Windows). Hence, you can install any additional software you may need to run your software. The disc space allocated to your virtual machine should be sufficient to store your software as well as any runtime data you need. Should you still need more disc space than is available, please send an e-mail with a short explanation how much disc space you need. You can start, stop, and restart you virtual machine both from within and via the TIRA web page.

\section{Access to Training Corpora on the Virtual Machines}

The training corpora of each shared task are mounted as a shared folder into your virtual machine.
On Ubuntu you can find the training data in \texttt{/media/training-data}. On Windows, the training data are found on drive~\texttt{F:$\backslash$}.
You have read-only access to the data sets which should be sufficient to test your software. Therefore, please \textbf{do not copy the training data into your virtual machine}.

\section{Completing a Software Submission}

You may move to complete a software submission once the test data is available on TIRA, and your software is deployed into your virtual machine. You can then self-evaluate your software on the test data by executing it via TIRA's shared task web page, using the test data as input.
